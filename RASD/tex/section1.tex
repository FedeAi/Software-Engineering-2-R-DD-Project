\section{Introduction}
eMall (e-Mobility for All) is an easy-to-use application which intent is to help the user to recharge their electric vehicle in order to reduce our carbon footprint.
Users need an application whose main intent is to plan the charging process of the electric vehicle, reducing the interference and constraints on our daily schedule.

\subsection{Purpose}

The aim of the product is to simplify the process of electric vehicle charging, improving the user experience of the customers.
The experience will be enhanced because many aspects of electric vehicle charging will be integrated and they will be located within a single service.
\\\\
Users will have the possibility to see which charging stations are available nearby and to also know the charging cost and if they have special offers.
They will also have the opportunity to book a charge at a station at a certain timeframe in advance and to start the charging process.
They will be notified when the process has finished and the car is fully charged, and they will be able to pay from the application.
Moreover, the eMall service has a smart system that suggets to the user the optimal charging solution, based on their schedule, the charging cost and the current state of charge.
\subsubsection{Goals}
\begin{enumerate}[label=$\bullet$ \textbf{G\arabic*:}]
        \item \textbf{Allow users to obtain information about nearby charging stations}
        \\
        The user can view information about any nearby charging station, 
        such as charging cost and about special offers, availability of every type of charging socket, 
         ******* and if a certain type of socket are all occupied,
         the estimated time for the first to be free.
        \item \textbf{Allow users to book a charge for a certain timeframe}
        \\
        The user can select any available charging station and book the charging for a certain timeslot. 
        \item \textbf{Allow users to start the charge}
        \\
        The user can remotely start the charge once the electric car has been connected to the station's socket. ********The booking of the station is not mandatory: if a station is free, the user is free to drive there and start the charge from the application.
        \item \textbf{Allow the users to know when the charging has finished}
        \\
        The user will be notified by the application when the charging of his vehicle has been completed.
        \item \textbf{Allow the users to pay for the charging service}
        \\
        The user has the option to pay directly from the eMall application for the obtained service.
        \item \textbf{Allow the users to receive suggestions on where to charge}
        \\
        The user can receive suggestions from eMall on the optimal station to charge, based on his schedule his vehicle's state of charge, the stations' prices and availability.
\end{enumerate}

\subsection{Scope}
\subsubsection{Phenomena}
\begin{table}[h]
        \centering
        \begin{tabular}{|c|c|c|}
        \hline
        Phenomenon                                        & Controller & Shared \\ \hline
        Users charge their vehicles at charging stations  & W          & N      \\ \hline
        User books a charge at a charging station         & W          & Y      \\ \hline
        WP3                                               & M          & Y      \\ \hline
        User registration                                 & M          & Y      \\ \hline
        User login                                        & W          & Y      \\ \hline
        User pays for the charge                          & W          & N      \\ \hline
        User recives advice on where to charge            & W          & Y      \\ \hline
        User is notified when the charge has been completed   & W          & Y      \\ \hline
        System notifies the user                          & M          & Y      \\ \hline
%        System determines the optimal station to charge   & M          & N      \\ \hline
        WP6                                               & W          & Y      \\ \hline
        WP6                                               & W          & Y      \\ \hline
        \end{tabular}
\end{table}

\subsection{Definitions, Acronyms, Abbrevations}
\subsection{Revision History}
\subsection{Reference Documents}
\subsection{Document Structure}